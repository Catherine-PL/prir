\documentclass[a4paper,12pt]{article}
\usepackage{amsmath}
\usepackage{amssymb}
\usepackage[polish]{babel}
\usepackage{polski}
\usepackage[utf8]{inputenc}
\usepackage{indentfirst}
\usepackage{geometry}
\usepackage{array}
\usepackage[pdftex]{color,graphicx}
\usepackage{subfigure}
\usepackage{afterpage}
\usepackage{setspace}
\usepackage{color}
\usepackage{wrapfig}
\usepackage{listings}
\usepackage{datetime}

\renewcommand{\onehalfspacing}{\setstretch{1.6}}

\geometry{tmargin=2.5cm,bmargin=2.5cm,lmargin=2.5cm,rmargin=2.5cm}
\setlength{\parindent}{1cm}
\setlength{\parskip}{0mm}

\newenvironment{lista}{
\begin{itemize}
  \setlength{\itemsep}{1pt}
  \setlength{\parskip}{0pt}
  \setlength{\parsep}{0pt}
}{\end{itemize}}

\newcommand{\linia}{\rule{\linewidth}{0.4mm}}

\definecolor{lbcolor}{rgb}{0.95,0.95,0.95}
\lstset{
  backgroundcolor=\color{lbcolor},
  tabsize=4,
  language=C++,
  captionpos=b,
  tabsize=3,
  frame=lines,
  numbers=left,
  numberstyle=\tiny,
  numbersep=5pt,
  breaklines=true,
  showstringspaces=false,
  basicstyle=\footnotesize,
  identifierstyle=\color{magenta},
  keywordstyle=\color[rgb]{0,0,1},
  commentstyle=\color[rgb]{0,0.5,0},
  stringstyle=\color{red}
}

\begin{document}

\noindent
\begin{tabular}{|c|p{11cm}|c|} \hline
Grupa~4 & Katarzyna Kosiak i~Michał Folwarski & \ddmmyyyydate\today \tabularnewline
\hline
\end{tabular}

\section*{Zadanie~5 -- Liczby pierwsze -- MPI}
Program ma za zadanie wykonać test pierwszości (sprawdzić czy dana liczba jest liczbą pierwszą) na zestawie liczb zapisanym w~pliku.
Algorytm sprawdzania testu pierwszości został zaimplementowany na bazie metody naiwnej oraz optymalizacji~o:
\begin{lista}
 \item pominięcie liczb parzystych,
 \item pominięcie liczb podzielnych przez trzy,
 \item sprawdzenie dzielników tylko do pierwiastka kwadratowego sprawdzanej liczby.
\end{lista}

\begin{lstlisting}
if (rank == 0) { //proces glowny
    if (numberOfProcesses == 1) {
        //wykonaj test pierwszosci dla wszystkich liczb
    } else {
        //rozdziel zadania
    }
} else { //pozostale procesy
    //wykonaj test pierwszosci dla otrzymanej liczby
}
\end{lstlisting}
Powyższy fragment kodu przedstawia główną ideę  w~jaki sposób zostało zorganizowane rozdzielanie zadań pomiędzy procesy w~standardzie Message Passing Interface (MPI). W~tym zadaniu równomierne rozdzielenie zadań pomiędzy dostępne procesy nie sprawdzi się, ponieważ weryfikowanie czy podana liczba jest pierwszą będzie wykonywało się w~różnym czasie. W~naszym przypadku jest to rozdzielone dynamicznie na jednoelementowe podzbiory -- każdy z~procesów (oprócz głównego) będzie otrzymywał po jednej liczbie do sprawdzenia. Dzięki takiemu podziałowi, dany proces otrzyma następne zadanie dopiero po zakończeniu poprzedniego. Zwiększanie ilości elementów w~podzbiorach powoduje, że otrzymane czasy obliczeń są nieznacznie większe lub mniejsze w~zależności od zestawu danych.

\begin{figure}[!hbp]
  \centering
    \includegraphics[width=0.6\textwidth]{chart}
  \caption{Wykres zależności czasu wykonania od ilości procesów}
  \label{chart-time-threads}
\end{figure}

Wykres na rysunku numer~1 przedstawia uśrednione wyniki czasu obliczeń dla zestawu pięciuset liczb pierwszych z~kilku prób na czterordzeniowym procesorze Intel\textregistered\ Core\texttrademark\ i7-920 z~technologią Hypher-Threading.

Jak widać na wykresie czas wykonania dla dwóch procesów jest większy niż dla jednego, ponieważ w~naszym rozwiązaniu pierwszy proces rozdziela zadania, a~pozostałe wykonują obliczenia -- czyli dla dwóch procesów tylko jeden z~nich wykonuje właściwe obliczenia. Dodatkowy czas w~stosunku do obliczeń na jednym procesie wynika z~tego, iż do zarządzania komunikacją pomiędzy procesami oraz przydziału im zadań, potrzebna jest pewna ilości zasobów (w~tym także obliczeniowych) oraz tym, że proces który wykona sprawdzenie danej liczby pod kątem pierwszości oczekuje pewien czas otrzymanie na kolejnej liczby do sprawdzenia.

Z~wykresu można też odczytać, że najlepszy czas obliczeń jest dla dziewięciu procesów (jeden proces rozdziela zadania, a~pozostałe osiem wykonują test pierwszości). Takie wyniku otrzymujemy dzięki temu, że procesor działa ma technologię Hypher-Threading oraz dlatego, że w~ciągu komunikacji są przerwy -- procesy obierają, przetwarzają i~wysyłają dane, a~następnie czekają na odebranie kolejnego zestawu danych. Planista systemowy zręcznie wykorzystuje to ``czekanie'' poprzez wykonanie tych zadań współbieżnie w~przeplocie. Dalsze zwiększanie liczby procesów (ponad dziewięć) nie przynosi wzrostu przyspieszenie, a~wyniki nieznacznie się różnią od tego najlepszego.

\begin{figure}[!htbp]
  \centering
    \includegraphics[width=0.6\textwidth]{chart2}
    \caption{Krzywa przyspieszenia obliczeń w~zależności od ilości procesów.}
  \label{chart-acceleration}
\end{figure}
Na rysunku numer~2 został przedstawiony wykres przyspieszenia obliczeń dla tego samego zestawu danych. Z~wykresu można odczytać, że największy przyrost szybkości obliczeń w~stosunku do obliczeń na jednym procesie jest około czterokrotny -- wyniki przyspieszenie w~dużej mierze zależą od zestawu danych i~od zaproponowanego sposobu podziału zadań pomiędzy procesami zaimplementowanego przez programistę.

\subsubsection*{Wnioski}
\noindent Dzięki zastosowaniu MPI, możliwe jest budowanie aplikacji do obliczeń równoległych za pomocą przesyłania komunikatów pomiędzy procesami. Sposób podziału zadań, wysyłania i~odbierania komunikatów należy wyłącznie do programisty -- niewątpliwie jest to trudniejsze niż w~przypadku korzystania z~OpenMP. W~tym zadaniu przetestowaliśmy możliwość dynamicznego rozdzielenia zadań pomiędzy wątki, a~co w~naszym przypadku przyniosło oczekiwane rezultaty -- obliczenia zostały zrównoleglone pomiędzy wątki i~dzięki temu zwiększyła się szybkość obliczeń.

\end{document}
